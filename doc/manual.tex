\documentclass[a4paper]{article}

\usepackage[utf8]{inputenc}

\title{Nidhugg -- Manual\\\large{Version %%NIDHUGGVERSION%%}}
\author{}
\date{}

\begin{document}

\maketitle

Nidhugg is a bug-finding tool which targets bugs caused by concurrency
and relaxed memory consistency in concurrent programs. It works on the
level of LLVM internal representation, which means that it can be used
for programs written in languages such as C or C++.

Currently Nidhugg supports the SC, TSO and PSO memory models. Target
programs should use pthreads for concurrency, and each thread should
be deterministic when run in isolation.

\pagebreak

\tableofcontents

\end{document}
