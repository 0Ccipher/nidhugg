\documentclass[a4paper]{article}

\usepackage[utf8]{inputenc}
\usepackage{amssymb}

\title{Nidhugg -- Manual\\\large{Version %%NIDHUGGVERSION%%}}
\author{}
\date{}

\newcommand{\limitsupport}[1]{\textbf{Supported only for #1.}}

\begin{document}

\maketitle

Nidhugg is a bug-finding tool which targets bugs caused by concurrency
and relaxed memory consistency in concurrent programs. It works on the
level of LLVM internal representation, which means that it can be used
for programs written in languages such as C or C++.

Currently Nidhugg supports the SC, TSO, PSO, POWER and ARM (partial) memory
models. Target programs should use pthreads for concurrency, and each
thread should be deterministic when run in isolation.

\pagebreak

\tableofcontents

\section{Stateless Model Checking with DPOR}

Stateless model checking systematically tests a program by running a
test case under many different thread schedules. This is done
systematically, in such a way that at least one execution is explored
for each equivalence class.

A test case must satisfy the following two conditions

\begin{description}
\item{\bf Finiteness}
%
  The test case must be finite in the following sense: There must be a
  bound $n\in\mathbb{N}$ such that all executions of the test case
  terminate within $n$ execution steps. (The user does not need to
  know the actual bound value.)
%
  If there is no such bound, you can still use Nidhugg to analyze the
  program, provided that you allow the tool to impose an artificial
  bound (see switches \texttt{--unroll} and
  \texttt{--max-search-depth}).
\item{\bf Determinism}
%
  The test case must be deterministic in the following sense: In a
  given state, executing a given execution step must always take the
  system to the same new state.
%
  This means that the test case may not e.g.\ check the time, or
  perform file I/O.
\end{description}

\section{Supported Memory Models}

Nidhugg supports the memory models SC, TSO, PSO, POWER and ARM.

The SC semantics are the classical interleaving semantics described
in~\cite{lamport1979make}. The TSO and PSO memory models, used on the
x86 and SPARC architectures, are described in~\cite{sparcv9}. For
POWER and ARM, we use the semantics described in~\cite{alglaveHerding}.

For SC, TSO and PSO, we use the analysis technique described
in~\cite{leonardssonSMCTSO}. The technique is precise, and guarantees
to explore all behaviors.

For POWER, we use the analysis technique described
in~\cite{leonardssonRSMCPOWER}. The technique is precise, and
guarantees to explore all behaviors.

For ARM, we use an adaptation of the technique
in~\cite{leonardssonRSMCPOWER}. The technique is an
\emph{under-approximation} of ARM, and does \emph{not} guarantee to
explore all behaviors.

\begin{thebibliography}{99}
\bibitem{lamport1979make}
  Lamport, Leslie. ``How to make a multiprocessor computer that correctly executes multiprocess programs.'' Computers, IEEE Transactions on 100.9 (1979): 690-691.
\bibitem{sparcv9}
  The SPARC architecture manual. Englewood Cliffs, NJ 07632: PTR Prentice Hall, 1994.
\bibitem{alglaveHerding}
  Alglave, Jade, Luc Maranget, and Michael Tautschnig. ``Herding cats: Modelling, simulation, testing, and data mining for weak memory.'' ACM Transactions on Programming Languages and Systems (TOPLAS) 36.2 (2014): 7.
\bibitem{leonardssonSMCTSO}
  Abdulla, Parosh Aziz, Stavros Aronis, Mohamed Faouzi Atig, Bengt Jonsson, Carl Leonardsson, Konstantinos Sagonas. ``Stateless model checking for TSO and PSO.'' Tools and Algorithms for the Construction and Analysis of Systems. Springer Berlin Heidelberg, 2015. 353--367.
\bibitem{leonardssonRSMCPOWER}
  Abdulla, Parosh Aziz, Mohamed Faouzi Atig, Bengt Jonsson, Carl Leonardsson. ``Stateless model checking for POWER.'' Pending peer review.
\end{thebibliography}

\section{Running Nidhugg}

The typical workflow of analyzing a test case with Nidhugg is as
follows:

\begin{enumerate}
\item Compile the source code into LLVM assembly using a compiler
  such as e.g.\ clang, clang++, llvm-gcc, etc.
\item Optionally use \textsf{nidhugg} to transform the assembly into
  new assembly that can be analyzed more efficiently.
\item Use \textsf{nidhugg} to analyze the LLVM assembly file.
\end{enumerate}

These steps are detailed in
Section~\ref{sec:using:nidhugg:directly}. For convenience, Nidhugg
provides a script \textsf{nidhuggc} which automates all the steps, for
the case when the test case is contained in a single C
file. Section~\ref{sec:using:nidhuggc} describes how to use
\textsf{nidhuggc}.

\subsection{Using Nidhugg Directly}\label{sec:using:nidhugg:directly}

\paragraph{Compilation}
%
Since Nidhugg works on LLVM internal representation, source code in
high-level languages such as e.g.\ C must be compiled before it can be
analyzed. This can be done with some compilers based on LLVM. C
programs are conveniently compiled with the \textsf{clang} compiler as
follows:

\vspace{5pt}
\noindent
\texttt{\$ clang $CFLAGS$ -emit-llvm -S -o $FILE$.ll $FILE$.c}

\vspace{5pt}\noindent
%
Here $CFLAGS$ can be any switches you would normally give to your
compiler when compiling \texttt{$FILE$.c}. Notice that if optimization
switches such as e.g.\ \texttt{-O2} are given to \textsf{clang}, then
the corresponding optimizations are performed on the code
\emph{before} analysis. This can be beneficial for analysis time
consumption, but may change the behavior of the program, e.g.\ in cases
where shared variables are not properly marked as \texttt{volatile}.

Make certain that the compiler supports the version of LLVM against
which Nidhugg is built. For \textsf{clang} this is easiest done by
making certain that the LLVM version is the same as the \textsf{clang}
version.

\paragraph{Multi-File Compilation}
%
To compile a test case which consists of multiple C files, the
procedure is similar to how you would normally compile the code: Use
e.g.\ \textsf{clang} to compile each C file into an object file, but
add switches \texttt{-emit-llvm -S} to generate the object file as
LLVM assembly. Then link the object files using \textsf{llvm-link}.

\vspace{5pt}
\noindent
\texttt{\$ clang -c $CFLAGS$ -emit-llvm -S -o $FILE_0$.ll $FILE_0$.c}\\
$\vdots$\\
\texttt{\$ clang -c $CFLAGS$ -emit-llvm -S -o $FILE_n$.ll $FILE_n$.c}\\
\texttt{\$ llvm-link -o $FILE$.ll $FILE_0$.ll $\cdots$ $FILE_n$.ll}

\paragraph{Transformation}
%
Before analyzing a test case with Nidhugg, the test case code can be
automatically rewritten in various ways, e.g.\ to improve analysis
performance or to put a bound on an infinite test case to make it
finite. This is done with a call to \textsf{nidhugg} as follows:

\vspace{5pt}
\noindent
\texttt{\$ nidhugg $TFLAGS$ --transform=$OUTFILE$.ll $INFILE$.ll}

\vspace{5pt}\noindent
%
Here we transform the code given as LLVM assembly in
\texttt{$INFILE$.ll}, and write the resulting code to the file
\texttt{$OUTFILE$.ll}. The switches given in $TFLAGS$ specify which
particular transformations should be performed. See
Section~\ref{sec:transform:switches} for details about the available
transformations.

\paragraph{Analysis}
%
A finite and deterministic test case in the form of LLVM assembly may
be analyzed by Nidhugg using a command like the following:

\vspace{5pt}
\noindent
\texttt{\$ nidhugg $FLAGS$ \{--sc,--tso,--pso,--power,--arm\} $FILE$.ll}

\vspace{5pt}\noindent
%
Additional switches are detailed in
Section~\ref{sec:analysis:switches}.
%
Nidhugg will systematically run executions of the test case, covering
all execution equivalence classes, under the given memory model. Then
it terminates, either giving an error trace or stating that no errors
were found.

\paragraph{Typical Usage}
%
For a small example of typical usage of Nidhugg, consider the C code
listed in Figure~\ref{fig:dekker:c}. We can analyze it under SC with
the following sequence of commands:

\vspace{5pt}
\noindent
\texttt{\$ clang -o test.ll -S -emit-llvm test.c}\\
\texttt{\$ nidhugg -sc test.ll}

\vspace{5pt}\noindent
%
Nidhugg will here tell us that no error was detected (the assert
statements are never violated). If we rerun the same test case under
TSO, Nidhugg will instead give us an error trace describing how the
assert statements may be violated under TSO. See
Section~\ref{sec:understand:error:traces} for details about how to
interpret the error trace.

In this case we performed no transformation. In the common case that
a test case contains (unbounded) loops, one may instead
want to use commands such as the following:

\vspace{5pt}
\noindent
\texttt{\$ clang -o test.ll -S -emit-llvm test.c}\\
\texttt{\$ nidhugg --unroll=10 --transform=test.trans.ll test.ll}\\
\texttt{\$ nidhugg -sc test.trans.ll}

\begin{figure}
\begin{verbatim}
// test.c
#include <pthread.h>
#include <assert.h>

volatile int x = 0, y = 0, c = 0;

void *thr1(void *arg){
  y = 1;
  // __asm__ volatile ("mfence" ::: "memory");
  if(!x){
    c = 1;
    assert(c == 1);
  }
  return NULL;
}

int main(int argc, char *argv[]){
  pthread_t t;
  pthread_create(&t,NULL,thr1,NULL);
  x = 1;
  // __asm__ volatile ("mfence" ::: "memory");
  if(!y){
    c = 0;
    assert(c == 0);
  }
  pthread_join(t,NULL);
  return 0;
}
\end{verbatim}
  \caption{Small Dekker test in C.}\label{fig:dekker:c}
\end{figure}

\subsubsection{Analysis Switches}\label{sec:analysis:switches}

\begin{description}
\item{\texttt{--sc/--tso/--pso/--power/--arm}}
%
  Analyze the test case under the SC, TSO, PSO, POWER or ARM memory
  model respectively.
\item{\texttt{--robustness}}
%
  Perform a robustness analysis. In addition to considering the usual
  safety criteria, also check robustness, and report an error if a
  non-robust execution is found.
%
  \limitsupport{SC, TSO, PSO}
\item{\texttt{--extfun-no-race=FUN}}
%
  Assume that the external function \texttt{FUN}, when called as
  blackbox, does not participate in any races. May be given several
  times. See Section~\ref{sec:stdlibc:races}.
%
  \limitsupport{SC, TSO, PSO}
\item{\texttt{--malloc-may-fail}}
%
  By default, when the test case calls \texttt{malloc}, Nidhugg will
  assume that the call succeeds, allocate memory and provide the
  resulting memory pointer as return value for the call. But, if
  \texttt{--malloc-may-fail} is given to Nidhugg, then the case when
  \texttt{malloc} fails and returns NULL is also analyzed.
%
  \limitsupport{SC, TSO, PSO}
\item{\texttt{--print-progress}}
%
  Continually print the number of hitherto analyzed executions to the
  terminal while running the analysis.
\item{\texttt{--print-progress-estimate}}
%
  Same as \texttt{--print-progress}, but also estimate the total
  number of executions of the test case, and correspondingly print a
  progress percentage together with the execution count.
\item{\texttt{--version}}
%
  Print the Nidhugg version and exit.
\end{description}

\subsubsection{Transformation Switches}\label{sec:transform:switches}

\begin{description}
\item{\texttt{--transform=$OUTFILE$}}
%
  Don't analyze the input test case. Instead run program
  transformations on it, and output the resulting code as LLVM
  assembly to the file $OUTFILE$. This switch should always be used
  for program transformation, and should be combined with switches
  that specify specific transformations.
\item{\texttt{--no-spin-assume}}
%
  Disable the \texttt{spin-assume} transformation. By default, the
  transformation will detect spin loops in the code, and replace them
  by \texttt{assume} statements. This transformation maintains
  satisfaction or dissatisfaction of the safety properties in the test
  case. However, this transformation does not in general maintain
  robustness violations.
\item{\texttt{--unroll=$N$}}
%
  Unroll all loops $N$ times, making the program loop free. A thread
  that attempts to execute a basic block from a loop more than $N$
  times in one go will be blocked. This is useful to make a finite
  test case out of an infinite one.

  Notice that this unrolling works on loops in the LLVM internal
  representation after compilation and possibly optimization. These
  loops do not necessarily correspond one-to-one with the loops in the
  high-level language.
\end{description}

\subsection{Using Nidhuggc for Single-File C Programs}\label{sec:using:nidhuggc}

In order to simplify the case where a test case consists of only a
single C file, Nidhugg provides a script \textsf{nidhuggc} which
automates the process described in
Section~\ref{sec:using:nidhugg:directly}. The script \textsf{nidhuggc}
can be used as follows:

\vspace{5pt}
\noindent
\texttt{\$ nidhuggc $CFLAGS$ -- \{--sc,--tso,--pso,--power,--arm\} $\backslash$}\\
\rule{20pt}{0pt}\texttt{$AFLAGS$ $TFLAGS$ $FILE$.c}

\vspace{5pt}\noindent
%
Here $CFLAGS$ are switches that will be given to the C compiler,
$TFLAGS$ are Nidhugg transformation switches, and $AFLAGS$ are Nidhugg
analysis switches. One common way to use \textsf{nidhuggc} is this
call:

\vspace{5pt}
\noindent
\texttt{\$ nidhuggc -O2 -- --tso --unroll=10 $FILE$.c}

\subsubsection{Switches}

\begin{description}
\item{\texttt{--version}}
%
  Print the Nidhugg version and exit.
\item{\texttt{--c}}
%
  Interpret the input file as C code regardless of its file name
  extension.
\item{\texttt{--cxx}}
%
  Interpret the input file as C++ code regardless of its file name
  extension.
\item{\texttt{--clang=$PATH$}}
%
  Specify the path to \textsf{clang}.
\item{\texttt{--clangxx=$PATH$}}
%
  Specify the path to \textsf{clang++}.
\item{\texttt{--nidhugg=$PATH$}}
%
  Specify the path to \textsf{nidhugg}.
\end{description}

\subsection{Understanding Error Traces}\label{sec:understand:error:traces}

In order to understand Nidhugg error traces, it is necessary to
understand the way Nidhugg identifies threads. Each thread is
associated with a thread ID (tid). The tid of the initial thread is
\texttt{<0>}, the threads that are created by \texttt{<0>} are named,
in order, \texttt{<0.0>}, \texttt{<0.1>}, \texttt{<0.2>} etc. The
threads created by \texttt{<0.1>} are named \texttt{<0.1.0>},
\texttt{<0.1.1>}, etc.

Since the operational semantics of the different memory models differ
significantly, the error traces are also different under different
models.

\subsubsection{Error Traces under TSO and PSO}

In addition to actual program threads, Nidhugg also reasons about
\emph{auxiliary threads}. These are hypothetical threads that carry
out non-deterministic events such as updates from a store buffer to
memory. The auxiliary threads belonging to a certain real thread
\texttt{<$i_0$.$\cdots$.$i_n$>} are named
\texttt{<$i_0$.$\cdots$.$i_n$/0>}, \texttt{<$i_0$.$\cdots$.$i_n$/1>},
etc. For example, in the case TSO, the auxiliary thread \texttt{<0/0>}
takes care of the memory updates from the store buffer of the initial
thread \texttt{<0>}.

Now we can consider the example error trace listed in
Figure~\ref{fig:ex:error:trace:dekker:tso}. It was produced by running the
following command on the test case given in Figure~\ref{fig:dekker:c}:

\vspace{5pt}
\noindent
\texttt{\$ nidhuggc -O2 -- --tso test.c}

\vspace{5pt}\noindent
%
Each line corresponds to an event, which is either the execution of an
instruction or a memory update. Each event is identified by a pair
\texttt{($tid$,$i$)}, where $tid$ is the thread ID of the executing
thread, and $i$ is the per-thread index of that event. Notice that
some events (e.g.\ \texttt{(<0>,2)}) are missing in the trace. The
missing events are events that are local, and do not affect any other
thread. Each event is annotated with the line of C code that produced
that instruction. Notice that one line of C code may correspond to
several instructions.

\begin{figure}
\begin{verbatim}
 Error detected:
  (<0>,1) test.c:16: int main(int argc, char *argv[]){
      (<0.0>,1) test.c:6: void *thr1(void *arg){
  (<0>,4) test.c:21: if(!y){
        (<0.0/0>,1) UPDATE test.c:7: y = 1;
  (<0>,5) test.c:21: if(!y){
  (<0>,8) test.c:23: assert(c == 0);
  (<0>,11) test.c:25: pthread_join(t,NULL);
      (<0.0>,2) test.c:9: if(!x){
    (<0/0>,1) UPDATE test.c:19: x = 1;
      (<0.0>,3) test.c:9: if(!x){
        (<0.0/0>,2) UPDATE test.c:10: c = 1;
    (<0/0>,2) UPDATE test.c:22: c = 0;
      (<0.0>,6) test.c:11: assert(c == 1);
                Error: Assertion violation at (<0.0>,9): (c == 1)
\end{verbatim}
  \caption{A TSO error trace for the program shown in
    Figure~\ref{fig:dekker:c}.}\label{fig:ex:error:trace:dekker:tso}
\end{figure}

We point out some main features of the error trace in
Figure~\ref{fig:ex:error:trace:dekker:tso}.
%
First we see that the load of \texttt{y} in event \texttt{(<0>,4)}
precedes the update to \texttt{y} by the other thread in event
\texttt{(<0.0/0>,1)}. Hence thread \texttt{<0>} sees the initial value
0 for \texttt{y}. Similarly the load \texttt{(<0.0>,2)} precedes the
update \texttt{(<0/0>,1)}, and so the other thread sees the value 0
for \texttt{x}.
%
At the end of the trace, we see that thread \texttt{<0>} updates
\texttt{c} with the value 0, just before thread \texttt{0.0} executes
\texttt{assert(c == 1)}, and so the assertion fails.

\subsubsection{Error Traces under POWER and ARM}

\begin{figure}
\begin{verbatim}
Error detected:
(<0>): Entering function main
(<0>,1): test.c:16 pthread_create(&t,NULL,thr1,NULL);
         store 0x0100000000000000 to [0x1ac39b0]
         store 0x01 to [0x1acd740]
  (<0.0>): Entering function thr1
(<0>,2): test.c:17 x = 1;
         store 0x01000000 to [0x1ad3110]
(<0>,3): test.c:19 if(!y){
         load 0x00000000 from [0x1ac0020] source: (initial value)
(<0>,4): test.c:20 c = 0;
         store 0x00000000 to [0x1acd680]
(<0>,5): test.c:21 assert(c == 0);
         load 0x00000000 from [0x1acd680] source: (<0>,4)
(<0>,6): test.c:23 pthread_join(t,NULL);
         load 0x0100000000000000 from [0x1ac39b0] source: (<0>,1)
  (<0.0>,1):
             load 0x01 from [0x1acd740] source: (<0>,1)
  (<0.0>,2): test.c:6 y = 1;
             store 0x01000000 to [0x1ac0020]
  (<0.0>,3): test.c:8 if(!x){
             load 0x00000000 from [0x1ad3110] source: (initial value)
  (<0.0>,4): test.c:9 c = 1;
             store 0x01000000 to [0x1acd680]
  (<0.0>,5): test.c:10 assert(c == 1);
             load 0x00000000 from [0x1acd680] source: (<0>,4)
  Error: Assertion failure at test.c:10: c == 1
\end{verbatim}
\caption{A POWER error trace for the program shown in Figure~\ref{fig:dekker:c}.}\label{fig:ex:error:trace:dekker:power}
\end{figure}

In the operational semantics of POWER and ARM, there are no auxiliary
threads or auxiliary events (as is the case under TSO and
PSO). Instead, each memory access is equipped with a parameter at the
time when it is committed. The parameter choice determines the
behavior of the event and its relation (memory ordering) to other
events. For stores, the parameter determines how the store is
coherence-ordered with other stores to the same memory location. For
loads, the parameter determines the source of the read value, i.e.,
the store which provided the observed value.

Now we can consider the example error trace given in
Figure~\ref{fig:ex:error:trace:dekker:power}. It was produced by
running the following command on the test case given in
Figure~\ref{fig:dekker:c}.

\noindent
\begin{verbatim}
$ nidhuggc -O2 -- --power test.c
\end{verbatim}

\noindent
Events represent the execution of a memory access. They are identified
as a pair \texttt{($tid$,$i$)} where $tid$ is the ID of the executing
thread, and $i$ is the index in program order of the executed memory
access instruction. Local instructions are not visible in the error
trace. For each event, the error trace contains the corresponding line
of source code, and an explanation of the behavior of the event when
it was committed.

For a store event, the explanation has the form
\texttt{store~$v$~to~[$a$]} where $v$ is the written value, and $a$ is
the memory address where it was stored. Notice that the value and
address are dependent on the architecture of the machine running
Nidhugg. For example, for event \texttt{(<0>,2)} in
Figure~\ref{fig:ex:error:trace:dekker:power}, the value 1 appears as
\texttt{0x01000000} because the test was run on a little endian
machine. The parameter (coherence order position) of the store event
is not printed.

For a load event, the explanation has the form
\texttt{load~$v$~from~[$a$]~source:~$s$} where $v$ is the observed
value that was read from memory address $a$. The parameter of the load
event is given as the source $s$. The source is the store event which
provided the observed value. In the general case, a load may have
multiple sources, in case the load reads multiple bytes written by
different stores.

A special case which is visible in
Figure~\ref{fig:ex:error:trace:dekker:power} is the event
\texttt{(<0>,1)}. This event, corresponding to the call to
\texttt{pthread\_create}, performs two stores: The thread ID of the
new thread is written to the output variable \texttt{t} here located
at \texttt{[0x1ac39b0]}. Furthermore, a byte-sized store is performed
to \texttt{[0x1acd740]}. The latter store is used to simulate
communication between threads during thread creation and thread
joining.

The error trace also contains lines specifying when the execution of
each thread enters and exits functions. Notice that the entering and
exiting of functions is related to when instructions contained in
those functions are \emph{fetched}. This may occur far earlier than
the fetched instructions are committed. Therefore it is not uncommon
or unexpected that the exit from a function appears earlier in the
error trace than the committing of its contained instructions.

For more details on the representation of (error) traces under POWER
and ARM, see~\cite{leonardssonRSMCPOWER}.

\begin{thebibliography}{99}
\bibitem{leonardssonRSMCPOWER}
  Abdulla, Parosh Aziz, Mohamed Faouzi Atig, Bengt Jonsson, Carl Leonardsson. ``Stateless model checking for POWER.'' Pending peer review.
\end{thebibliography}

\section{Compatibility: Pthreads, stdlib, etc.}

\subsection{Pthreads}

The following pthread functions are currently supported by Nidhugg:

\begin{description}
\item{\texttt{pthread\_create}}
\item{\texttt{pthread\_join}}
\item{\texttt{pthread\_exit}}
\item{\texttt{pthread\_mutex\_init}}
\item{\texttt{pthread\_mutex\_lock}}
\item{\texttt{pthread\_mutex\_unlock}}
\item{\texttt{pthread\_mutex\_destroy}}
\end{description}

\paragraph{The following are only supported under SC, TSO and PSO.}

\begin{description}
\item{\texttt{pthread\_self}}
\item{\texttt{pthread\_mutex\_trylock}}
\item{\texttt{pthread\_cond\_init}}
\item{\texttt{pthread\_cond\_signal}}
\item{\texttt{pthread\_cond\_broadcast}}
\item{\texttt{pthread\_cond\_wait}}
\item{\texttt{pthread\_cond\_destroy}}
\end{description}

\subsection{The Standard C library}\label{sec:stdlibc}

Nidhugg has limited support for functions in the standard C
library. The functions listed below have custom support. Other
functions in the standard library can, under SC, TSO and PSO, be
called as blackboxes. See Section~\ref{sec:external:functions}.

\paragraph{Functions with custom support:}

\begin{description}
\item{\texttt{malloc}}
\item{\texttt{free}}
\item{\texttt{atexit}} (\limitsupport{SC, TSO, PSO})
\item{\texttt{\_\_assert\_fail}} (Called by the \texttt{assert} macro
  in \textsf{assert.h}.)
\end{description}

\subsection{External Functions (SC, TSO, PSO only)}\label{sec:external:functions}

While running an execution, if Nidhugg encounters an unknown external
function (e.g.\ some function from the standard C library), it will
handle it by printing a warning and making \emph{an actual call} to
the function. This will usually work fine for well-behaved functions
such as e.g.\ the ones in \textsf{string.h}. But may cause undefined
behavior in other cases, such as e.g.\ when calling functions that deal
with signal handling or file I/O.

\begin{center}
  \begin{tabular}{|p{.8\linewidth}|}
    \hline
    \multicolumn{1}{|c|}{\textbf{Warning}}\\
    \hline
    Notice that the external functions are actually run on your system. If
    the analyzed code does undesirable things to the system, those things
    will really be executed.\\
    \hline
  \end{tabular}
\end{center}

\subsubsection{External Functions and Races}\label{sec:stdlibc:races}

Since external functions are called as blackboxes, there is no way for
Nidhugg to know which memory locations will be accessed by the
functions, which mutexes will be locked or unlocked, etc. In order to
provide a complete under-approximation, Nidhugg therefore assumes that
external functions conflict with \emph{all} other accesses to memory,
mutexes, etc.
%
This may cause Nidhugg to explore a huge number of superfluous
computations, which differ only in the order between unrelated or
irrelevant external function calls.
%
To avoid this problem, the user may use the switch
\texttt{--extfun-no-race} to specify external functions which should
be assumed to not participate in races.

For example, consider the program given in
Figure~\ref{fig:printf:prog}. It has two threads, each calling
\texttt{printf} once. There are no (real) data
races. Figure~\ref{fig:printf:run:16} shows analysis of the program
using \textsf{nidhuggc}. Here the calls to \texttt{printf} are assumed
to race both with each other, and with the loads of \texttt{t0} and
\texttt{t1} in the main thread.
%
Nidhugg explores computations corresponding to all the different ways
to resolve the races. In total 16 computations are explored. One
single computation would suffice.
%
In Figure~\ref{fig:printf:run:1} we show how to use the switch
\texttt{--extfun-no-race} to fix the problem, and correctly explore
only one computation.

\begin{figure}
\begin{verbatim}
// testprintf.c

#include <pthread.h>
#include <stdio.h>

void *p(void *arg){
  printf("foo\n");
  return NULL;
}

int main(int argc, char *argv[]){
  pthread_t t0, t1;
  pthread_create(&t0,NULL,p,NULL);
  pthread_create(&t1,NULL,p,NULL);
  pthread_join(t0,NULL);
  pthread_join(t1,NULL);
  return 0;
}
\end{verbatim}
\caption{A small race free program.}\label{fig:printf:prog}
\end{figure}

\begin{figure}
  \scriptsize{
\begin{verbatim}
$ nidhuggc --sc testprintf.c
* Nidhuggc: $ clang -o /tmp/tmp4ndgl1_l/tmpjs60nht_.ll -S -emit-llvm -g testprintf.c
* Nidhuggc: $ /usr/bin/nidhugg --sc /tmp/tmp4ndgl1_l/tmpjs60nht_.ll
WARNING: Calling unknown external function printf as blackbox.
foo
foo
foo
foo
foo
foo
foo
foo
foo
foo
foo
foo
foo
foo
foo
foo
foo
foo
foo
foo
foo
foo
foo
foo
foo
foo
foo
foo
foo
foo
foo
foo
Trace count: 16 (also 0 sleepset blocked)
No errors were detected.
Total wall-clock time: 0.08 s
\end{verbatim}}
\caption{Analysis of the program in Figure~\ref{fig:printf:prog}. Here
  \texttt{printf} is assumed to race with any memory access. This
  causes a wasteful exploration of 16 different
  computations.}\label{fig:printf:run:16}
\end{figure}

\begin{figure}
  \scriptsize{
\begin{verbatim}
$ nidhuggc --sc --extfun-no-race=printf testprintf.c
* Nidhuggc: $ clang -o /tmp/tmpsqqgz4fz/tmpabpqpy9z.ll -S -emit-llvm -g testprintf.c
* Nidhuggc: $ /usr/bin/nidhugg --sc --extfun-no-race=printf /tmp/tmpsqqgz4fz/tmpabpqpy9z.ll
WARNING: Calling unknown external function printf as blackbox.
foo
foo
Trace count: 1 (also 0 sleepset blocked)
No errors were detected.
Total wall-clock time: 0.07 s
\end{verbatim}}
\caption{Analysis of the program in Figure~\ref{fig:printf:prog} using
  the switch \texttt{--extfun-no-race}. As intended, only one
  computation is explored.}\label{fig:printf:run:1}
\end{figure}


\subsection{Verifier Functions}

Nidhugg also provides a number of functions that can be convenient for
modelling and verification. Declare the functions in your C code
before use, making sure to use the same signatures as described here.

\begin{description}
\item{\texttt{void \_\_VERIFIER\_assume(int b)}}
%
  This function checks the value of \texttt{b}. If $\texttt{b} = 0$,
  the execution blocks indefinitely, otherwise the call does nothing.
\item{\texttt{int \_\_VERIFIER\_nondet\_int()}}
%
  Returns a non-deterministic integer value. The function gives
  support for a very limited non-determinism. Each call to
  \texttt{\_\_VERIFIER\_nondet\_int} will return a value which is
  unpredictable before the analysis. But a given call to
  \texttt{\_\_VERIFIER\_nondet\_int} by a given thread will always
  return the same value in all executions that are explored.
%
  \limitsupport{SC, TSO, PSO}
\item{\texttt{unsigned int \_\_VERIFIER\_nondet\_uint()}}
%
  Same as \texttt{\_\_VERIFIER\_nondet\_int}.
%
  \limitsupport{SC, TSO, PSO}
\item{\texttt{\_\_VERIFIER\_atomic\_*}}
%
  Any function, defined by the user, whose name starts with
  \texttt{\_\_VERIFIER\_atomic\_} will execute atomically. This also
  means that it will execute under sequential consistency, and act as
  a full memory fence.
%
  \limitsupport{SC, TSO, PSO}
\end{description}

\subsection{Fences and Intrinsics}

The following are currently supported memory fences:

\begin{description}
\item{\texttt{\_\_asm\_\_ volatile ("mfence" ::: "memory")}} Full
  memory fence. Supported under SC (with no effect), TSO and PSO.
\item{\texttt{\_\_asm\_\_ volatile ("sync" ::: "memory")}} POWER \textsf{sync} fence. See~\cite{alglaveHerding}.
\item{\texttt{\_\_asm\_\_ volatile ("lwsync" ::: "memory")}} POWER \textsf{lwsync} fence. See~\cite{alglaveHerding}.
\item{\texttt{\_\_asm\_\_ volatile ("eieio" ::: "memory")}} POWER \textsf{eieio} fence. See~\cite{alglaveHerding}.
\item{\texttt{\_\_asm\_\_ volatile ("isync" ::: "memory")}} POWER \textsf{isync} fence. See~\cite{alglaveHerding}.
\item{\texttt{\_\_asm\_\_ volatile ("dmb" ::: "memory")}} ARM \textsf{dmb} fence. See~\cite{alglaveHerding}.
\item{\texttt{\_\_asm\_\_ volatile ("dsb" ::: "memory")}} ARM \textsf{dsb} fence. See~\cite{alglaveHerding}.
\item{\texttt{\_\_asm\_\_ volatile ("isb" ::: "memory")}} ARM \textsf{isb} fence. See~\cite{alglaveHerding}.
\item{\texttt{\_\_sync\_synchronize()}} Full memory fence. \limitsupport{SC, TSO, PSO}
\item{\texttt{\_\_atomic\_thread\_fence(\_\_ATOMIC\_SEQ\_CST)}} Full memory fence. \limitsupport{SC, TSO, PSO}
\end{description}

Built-in atomic functions provided by the compiler are typically
supported under SC, TSO and PSO, but only for the model
\texttt{\_\_ATOMIC\_SEQ\_CST}. This allows use of e.g.\ compare and
exchange (\texttt{\_\_atomic\_compare\_exchange\_n}) or atomic
increase (\texttt{\_\_atomic\_add\_fetch}).

\begin{thebibliography}{99}
\bibitem{alglaveHerding}
  Alglave, Jade, Luc Maranget, and Michael Tautschnig. ``Herding cats: Modelling, simulation, testing, and data mining for weak memory.'' ACM Transactions on Programming Languages and Systems (TOPLAS) 36.2 (2014): 7.
\end{thebibliography}

\end{document}
